\documentclass[english]{article}
\usepackage[T1]{fontenc}
\usepackage[utf8]{inputenc}

\makeatletter
\makeatother

\usepackage{babel}

\begin{document}

\title{Informe HCI TPE2}

\maketitle

\section{Decisiones de diseño}


\subsection*{Vista de Categorías, Subcategorías y productos }

Si bien se trató de ser lo mas fiel posible al diseño original, se
hicieron cambios leves.

En cuanto a las listas, se optó por: 
\begin{itemize}
\item Agregar el precio a cada producto, ya que este dato es lo primero
que el usuario mira al momento de tomar una decisión. 
\item Usar una tipografía mas grande. Cuando se diseñó el menú, este aparentaba
ser mas {}``agradable” cuando cada ítem utilizaba una única línea,
pero al ver los resultados obtenidos en la práctica, se decidió que
era un buen cambio. Además, no se había tenido en cuenta a los usuarios
con problemas de visión (hay que tener presente que estos son dispositivos
con pantallas muy reducidas y que además uno puede encontrarse en
ambientes muy iluminados y no verse bien el brillo de la pantalla)
ni a los usuarios con dedos grandes, a los que puede dificultarse
seleccionar un ítem de la lista debido a su tamaño reducido. 
\item Se agregó un texto en la parte inferior de la pantalla que muestra
por unos segundos la opción seleccionada (para complementar al breadcrumb
que puede no ser notado debido a su tamaño).\\

\end{itemize}
En cuanto a su implementación, por experiencia con el trabajo práctico
anterior, se implemento un \textquotedbl{}$category$ $manager$\textquotedbl{},
que se encarga de almacenar una lista de categorías y subcategorías
que ya fueron pedidas al servidor y de esta manera, se reduce a uno
la cantidad de peticiones de cada categoría. Consideramos esto un
detalle muy importante a tener en cuenta ya que (por lo general),
los dispositivos móviles poseen conexiones a internet mediante 3g
(conexiones muy inferiores a las que se tienen en el hogar) y que
además es cobradas por paquetes. Por lo que el uso del programa terminaría
saliendole muy caro al cliente si va y viene constantemente en la lista.\\
 Se consideró hacer lo mismo para el listado de productos, pero
se presentaba el problema que pueden llegar a existir cientos de productos
en el servidor y esto resultar en un programa final mas lento (debido
a demasiada información cargada en memoria%
\footnote{Se tuvo en cuenta el hecho de usar una lista de tamaño fijo en la
que se guarden los últimos elementos consultados y los demás sean
descartados, pero se dejó esto para una futura implementación ya que
se tenían otras prioridades de funcionalidad mas importantes.%
}).\\
 

En cuanto a la vista de producto, se logró implementar una pantalla
muy fiel a la planeada. Un detalle muy importante que hay que resaltar
es el hecho que al momento de listar los datos del producto, se usa
la misma tipografía para la etiqueta y su valor, esto se vería muchísimo
mejor si fuese como en la vista de información de ordenes.\\



\subsection*{Filtro, paginación y ordenamiento de la información }

Al principio del trabajo se pensó en implementar una barra de búsqueda
en la parte superior de la pantalla. Sin embargo, al momento de comenzar
a programar, se encontró que android ya ofrecía una de estas (y de
hecho era casi igual a la pensada originalmente!). Aunque por falta
de tiempo, no se pudo llegar a implementar esta funcionalidad ni la
de paginación (load on demand).

En cuanto al filtro, éste si fue aplicado. En cualquier momento se
puede ingresar texto y la lista se fija si hay algún producto que
$contenga$ las palabras ingresadas. Como no es para nada intuitivo
el ingresar un texto sin que halla ningún lugar para escribir (solo
seleccionar) se optó inicialmente por crear un menú $ayuda$ en el
menú principal en donde se hable sobre esto (y posiblemente sobre
otras cosas mas). Pero, por experiencia, se sabe que el usuario no
ingresará en este menu a menos que le sea imprescindible. Por lo que
se tuvo que pensar otra solución, ya que no se quiere ofrecer funcionalidad
que no se use por no ser {}``visible''.

La solución que actualmente se encuentra aplicada es la de mostrar
un texto corto explicando la funcionalidad mencionada en un $Toast$
por un corto intervalo de tiempo. Por supuesto, en un lugar poco invasivo
en la pantalla.\\



\subsection*{Menú principal}

Para simplicidad de la aplicación y mejor aprovechamiento del espacio
en la pantalla, se trató de reducir al mínimo la cantidad de objetos
a mostrar. Actualmente (y debido a la cantidad de funcionalidad) solo
se tienen tres botones. Uno para ver productos, otro para ver ordenes
existentes y un botón para desloguearse.

Por experiencia en trabajos anteriores, se sabe que no es una buena
práctica, habilitar botones al usuario cuando en realidad no tiene
sentido debido al estado en el que la aplicación se encuentra. Y es
por esto que el botón de ver ordenes se desactiva automáticamente
cuando el servidor indica que el usuario actual no posee ordenes.\\



\section{Conclusiones}

Luego de concluido el trabajo práctico, nos dimos cuenta sobre el
potencial de $Android$, lo fácil que es diseñar aplicaciones con
esta herramienta (sin mencionar el complicado arranque desde cero)
y que nos pareció una idea muy innovadora que se halla implementado
el uso de esta plataforma en el curso .

No esta de más mencionar que nos hubiese resultado muy interesante
poder probar nuestro resultado final en un aparato real ya que después
de todo es donde esta destinada a andar.
\end{document}

