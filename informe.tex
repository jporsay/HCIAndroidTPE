% Preview source code

%% LyX 1.6.5 created this file.  For more info, see http://www.lyx.org/.
%% Do not edit unless you really know what you are doing.
\documentclass[english]{article}
\usepackage[T1]{fontenc}
\usepackage[latin9]{inputenc}
\usepackage{babel}

\begin{document}

\title{Informe HCI TPE2}

\maketitle

\section{Decisiones de diseño}


\subsection*{Vista de Categorías, Subcategorías y productos }

Si bien se trató de ser lo mas fiel posible al diseño original, se
hizieron cambios leves.

En cuanto a las listas, se optó por:
\begin{itemize}
\item Agregar el precio a cada producto, ya que este dato es lo primero
que el usuario mira al momento de tomar una decisión.
\item Usar una tipografia mas grande. Cuando se diseñó el menú, este aparentaba
ser mas \textquotedblleft{}agradable\textquotedblright{} cuando cada
item utilizaba una única línea, pero al ver los resultados obtenidos
en la práctica, se decidió que era un buen cambio. Además, no se había
tenido en cuenta a los usuarios con problemas de visión (hay que tener
presente que estos son dispositivos con pantallas muy reducidas y
que además uno puede encontrarse en ambientes muy iluminados y no
verse bien el brillo de la pantalla).
\item Se agregó un texto en la parte inferior de la pantalla que muestra
por unos segundos la opción seleccionada (para complementar al breadcrumb
que puede no prestarsele mucha atención debido a su tamaño).\\

\end{itemize}
En cuanto a la vista de producto, se logró implementar una pantalla
muy fiel a la planeada. Un detalle muy importante que hay que resaltar
es el hecho que al momento de listar los datos del producto, se usa
la misma tipografía para la etiqueta y su valor, esto se vería muchisimo
mejor si fuese como en la vista de información de ordenes. 


\subsection*{Filtro, paginación y ordenamiento de la información }

Al principio del trabajo se pensó en implementar una barra de búsqueda
en la parte superior de la pantalla. Sin embargo, al momento de comenzar
a programar, se encontró que android ya ofrecía una de estas (y de
hecho era casi igual a la pensada originalmente!). Aunque por falta
de tiempo, no se pudo llegar a implementar esta funcionalidad ni la
de paginación (load on demand). 

En cuanto al filtro, este si fue aplicado y en cualquier momento se
puede ingresar texto y la lista se fija si hay algun producto que
$contenga$ las palabras ingresadas. Como no es para nada intuitivo
el ingresar un texto sin que halla ningun lugar para escribir (solo
seleccionar) se optó inicualmente por crear un menu $ayuda$ en el
menú principal en donde se hable sobre esto (y posiblemente sobre
otras cosas mas). Pero, por experiencia, se sabe que el usuario no
ingresará en este menu a menos que le sea imprescindible. Por lo que
se tuvo que pensar otra solucion, ya que no se quiere ofrecer funcionalidad
que no se use devido a la que no es {}``visible''. 

La solución que se actualmenteencuentra aplicada es la de mostrar
un texto volando por un corto intervalo de tiempo, en un lugar poco
invasivo en la pantalla y por supuesto, no modal.


\subsection*{Menú principal}

Para simplicidad de la aplicación y mejor aprovechamiento del espacio
en la pantalla, se trató de reducir al minimo la cantidad de objetos
a mostrar. Actualmente (y debido a la cantidad de funcionalidad) solo
se tienen tres botones. Uno para ver productos y otro para ver ordenes
existentes y por supuesto, botón para desloguearse. 

Por experiencia en trabajos anteriores, se sabe que no es una buena
práctica, habilitar botones al usuario cuando en realidad no tiene
sentido debido al estado en el que la aplciacipon se encuentra. Y
es por esto que el botón de ver oredenes se desactiva automaticamente
cuando el servidor indica que el usuario actual no posee ordenes.


\section{Conculsiones}

Luego de concluido el trabajo práctico, nos dimos cuenta sobre el
potencial de $Andorid$ y lo fácil que es diseñar aplicaciones con
esta herramienta (no esta de más mencionar el complicado arranque
desde $0$). 

No esta de mas mencionar que nos hubiese resultado muy intereante
poder probar nuestro resultado final en un aparato real ya que después
de todo es donde esta destinada a andar.
\end{document}

